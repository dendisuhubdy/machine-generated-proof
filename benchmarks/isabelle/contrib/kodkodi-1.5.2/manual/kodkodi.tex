\documentclass[a4paper,12pt]{article}
\usepackage[T1]{fontenc}
\usepackage{amsmath}
\usepackage{amssymb}
\usepackage{bigstrut}
\usepackage{color}
\usepackage{mathpazo}
\usepackage{multicol}
\usepackage{multirow}
\usepackage{supertabular}
\usepackage{url}
\usepackage[scaled=.85]{beramono}

\oddsidemargin=4.6mm
\evensidemargin=4.6mm
\textwidth=150mm
\topmargin=4.6mm
\headheight=0mm
\headsep=0mm
\textheight=234mm

\begin{document}

\title{Clubbing Cods \\[\smallskipamount]
\Large A User's Guide to Kodkodi 1.5.2}
\author{Jasmin Christian Blanchette \\
{\normalsize Institut f\"ur Informatik, Technische Universit\"at M\"unchen}}

\maketitle

\tableofcontents

\setlength{\parskip}{.7em plus .2em minus .1em}
\setlength{\parindent}{0pt}
\setlength{\abovedisplayskip}{\parskip}
\setlength{\abovedisplayshortskip}{.9\parskip}
\setlength{\belowdisplayskip}{\parskip}
\setlength{\belowdisplayshortskip}{.9\parskip}

% General-purpose enum environment with correct spacing
\newenvironment{enum}%
    {\begin{list}{}{%
        \setlength{\topsep}{.1\parskip}%
        \setlength{\partopsep}{.1\parskip}%
        \setlength{\itemsep}{\parskip}%
        \advance\itemsep by-\parsep}}
    {\end{list}}

\def\pre{\begingroup\vskip0pt plus1ex\advance\leftskip by\leftmargin
\advance\rightskip by\leftmargin}
\def\post{\vskip0pt plus1ex\endgroup}

\def\prew{\pre\advance\rightskip by-\leftmargin}
\def\postw{\post}

\def\cpp{C\nobreak\raisebox{.1ex}{+}\nobreak\raisebox{.1ex}{+}}

\section{Introduction}
\label{introduction}

Kodkodi is a front-end for the Java library Kodkod \cite{torlak-jackson-2007}, a
highly-optimized relational model finder developed by the Software Design Group
at MIT. Kodkod is based on SAT solving and supports first-order logic with
relations, transitive closure, and partial instances. Kodkod forms the basis of
version 4 of the Alloy Analyzer \cite{jackson-2006}. The Kodkodi
front-end is designed to make the Kodkod library available to other programming
languages than Java.

\newbox\boxA
\setbox\boxA=\hbox{\texttt{NOSPAM}}

This manual explains the concrete syntax supported by Kodkodi. It also explains
how to install the tool on your workstation. If you use Kodkodi in conjunction
with Nitpick for Isabelle/HOL \cite{nitpick-2009}, read the installation
instructions in the Nitpick manual \cite{blanchette-2009}.

Comments and bug reports concerning Kodkodi or this manual should be directed to
\texttt{blan{\color{white}NOSPAM}\kern-\wd\boxA{}chette@\allowbreak
in.\allowbreak tum.\allowbreak de}.

%\vskip2.5\smallskipamount
%
%\textbf{Acknowledgment.} The author would like to thank Mark Summerfield for
%suggesting several textual improvements.

\section{Installing the Tool}
\label{installing-the-tool}

To install Kodkodi, download and extract the archive
\url{http://isabelle.in.tum.de/~blanchet/kodkodi-1.5.0.tgz}. The \texttt{.jar}
files for Kodkodi, the Kodkod library, the portable SAT4J solver, and the ANTLR
3.1.1 runtimes are located in the \texttt{jar} subdirectory. Kodkodi
requires a Java 1.6 virtual machine, normally called \texttt{java}. To run
Kodkodi, you must add the \texttt{.jar} files in this directory to the Java
classpath (e.g., by setting the \texttt{CLASSPATH} environment variable) and
execute

\pre
\ttfamily\small java de.tum.in.isabelle.Kodkodi.Kodkodi
\post

To recompile Kodkodi, you need a Java compiler and the ANTLR 3.1.1 parser
generator tools.

For better performance, it is recommended to install a C or \cpp{} SAT solver.
Kodkodi already includes binary versions of MiniSat, CryptoMiniSat, and
Lingeling for various platforms in the \texttt{jni} subdirectory. Follow the
instructions on Kodkod's home page \cite{kodkod-2009} to install other SAT
solvers integrated using the JNI, or install a command-line solver of your
choice and specify it using the \texttt{External} or \texttt{ExternalV2}
described in \S\ref{kodkod-options}.

\section{First Steps}
\label{first-steps}

Kodkodi takes its input from standard input and writes its output to standard
output (on success) or standard error (on failure). Examples are provided in the
\texttt{examples} directory. When invoked with wrong command-line arguments,
Kodkodi displays the usage text:
%
\pre
\ttfamily\small
Usage: java de.tum.in.isabelle.Kodkodi.Kodkodi [options] \\
options: \\
\hbox{}~~-help~~~~~~~~~~~~~~~Show usage and exit \\
\hbox{}~~-verbose~~~~~~~~~~~~Produce more output \\
\hbox{}~~-solve-all~~~~~~~~~~Output all solutions for each problem \\
\hbox{}~~-prove~~~~~~~~~~~~~~Output minimal unsatisfiable core \\
\hbox{}~~-max-solutions~~~~~~Maximum number of solutions to generate (default:~$\infty$) \\
\hbox{}~~-clean-up-inst~~~~~~Remove trivial parts of instance from output \\
\hbox{}~~-max-msecs <num>~~~~Maximum running time in milliseconds \\
\hbox{}~~-max-threads <num>~~Maximum number of simultaneous threads \\ \hbox{}~~~~~~~~~~~~~~~~~~~~~~(default:~$\left<\textit{machine-dependent-value}\right>$) \\
\hbox{}~~-server~~~~~~~~~~~~~Run as TCP server \\
\hbox{}~~-port <number>~~~~~~Listen to specified port (default:~9128)
\post

Example input files are provided in the \texttt{examples} directory:

\pre
\ttfamily\small
\$ \textbf{export JAVA\_LIBRARY\_PATH=jni/x86-linux:\$JAVA\_LIBRARY\_PATH} \\
\$ \textbf{de.tum.in.isabelle.Kodkodi.Kodkodi < examples/pigeonhole.kki} \\
\slshape *** PROBLEM 1 *** \\[2\smallskipamount]
---OUTCOME--- \\
UNSATISFIABLE \\[2\smallskipamount]
---STATS--- \\
p cnf 54 68 \\
primary variables:~6 \\
parsing time:~65 ms \\
translation time:~92 ms \\
solving time:~0 ms
\post

(Replace \texttt{x86-linux} with the appropriate platform.)

The output is to be interpreted as follows:
%
\begin{itemize}
\item The two numbers following \texttt{p~cnf} indicate the number of
propositional variables and CNF clauses.
\item The primary variables are those variables that encode Kodkod relations.
(The other variables are auxiliary variables.)
\item The time spent on a problem is divided between parsing time (the time
Kodkodi took to convert the text file into a Kodkod abstract syntax tree),
translation time (the time Kodkod took to convert the abstract syntax tree into
a SAT problem), and solving time (the time the SAT solver took to solve the
problem).
\end{itemize}
%
To obtain all the solutions instead of only the first solution to a problem or
set of problems, pass the \texttt{-solve-all} option. To obtain only the first
$n$~solutions, pass \texttt{-max-solutions~$n$}.

\section{Input Format}
\label{input-format}

Kodkodi's input format is modeled after the output format of the
\texttt{toString()} implementations found in the Kodkod library. The operator
that are available in Alloy 4 are given the same precedences as they have there.

The grammar is expressed using a variant of Extended Backus-Naur Form. The
actual grammar used by Kodkodi is written using ANTLR and can be found in the
file \texttt{src/Kodkodi.g}.

\subsection{Lexical Issues}
\label{lexical-issues}

%%% TODO: ADD C comments
The grammar is based on the following lexical units, or tokens:
%
$$\small\begin{aligned}
\textit{WHITESPACE} & \,::=\, (\verb*| | \mid \textbf{\char`\\n} \mid \textbf{\char`\\r} \mid \textbf{\char`\\t} \mid \textbf{\char`\\v})^+ \\
\textit{INLINE\_COMMENT} & \,::=\, \verb|//|\, {\sim}(\textbf{\char`\\n})^*\,(\textbf{\char`\\n} \mid \textbf{eof}) \\
\textit{BLOCK\_COMMENT} & \,::=\, \verb|/*| .^+ \verb|*/| \\
\textit{NUM} & \,::=\, [\verb|+| \mid \verb|-|]\, (\verb|0| \mid \verb|1| \mid \ldots \mid \verb|9|)^+ \\
\textit{STR\_LITERAL} & \,::=\, \verb|"|\, {\sim}(\verb|"| \mid \textbf{\char`\\n})^*\, \verb|"| \\
\textit{ATOM\_NAME} & \,::=\, \verb|A|\; \textit{NAT} \\
\textit{UNIV\_NAME} & \,::=\, \verb|u|\; \textit{NAT} \\
\textit{OFF\_UNIV\_NAME} & \,::=\, \verb|u|\; \textit{NAT}\; @\; \textit{NAT} \\
\textit{TUPLE\_NAME} & \,::=\, (\verb|P| \mid \verb|T|\; \textit{NAT}\; \verb|_|)\, \textit{NAT} \\
\textit{RELATION\_NAME} & \,::=\, (\verb|s| \mid \verb|r| \mid \verb|m|\; \textit{NAT}\; \verb|_|) \,\textit{NAT} \\
\textit{VARIABLE\_NAME} & \,::=\, (\verb|S| \mid \verb|R| \mid \verb|M|\; \textit{NAT}\; \verb|_|) \,\textit{NAT} \;\verb|'|^? \\
\textit{TUPLE\_REG} & \,::=\, \verb|$|\, (\verb|A| \mid \verb|P| \mid \verb|T|\; \textit{NAT}\; \verb|_|) \,\textit{NAT} \\
\textit{TUPLE\_SET\_REG} & \,::=\, \verb|$|\, (\verb|a| \mid \verb|p| \mid \verb|t|\; \textit{NAT}\; \verb|_|) \,\textit{NAT} \\
\textit{FORMULA\_REG} & \,::=\, \verb|$f|\; \textit{NAT} \\
\textit{REL\_EXPR\_REG} & \,::=\, \verb|$e|\; \textit{NAT} \\
\textit{INT\_EXPR\_REG} & \,::=\, \verb|$i|\; \textit{NAT}
\end{aligned}$$
%
\textit{NAT} abbreviates $\verb|0| \mid (\verb|1| \mid \ldots \mid \verb|9|)\,
(\verb|0| \mid \ldots \mid \verb|9|)^*$.
%
Whitespace and comments are ignored, except as token separators. In addition to
the tokens listed above, various keywords and operators are recognized as
tokens. These are shown in bold.

The table below describes the lexical conventions adopted for naming atoms,
tuples, relations, variables, and registers.

\begin{center}
\small
\begin{supertabular}{l|c|p{89mm}} %% TYPESETTING
Token Name & Syntax & Description \\[.4ex]
\hline
\strut\vphantom{$A^{B^{C^{D}}}$}%
\textit{ATOM\_NAME} & \texttt{A}$j$ & Atom at index $j$ in the universe\bigstrut \\
\textit{UNIV\_NAME} & \texttt{u}$n$ & Set of atoms $\{\texttt{A0}, \ldots, \texttt{A}(n - 1)\}$\bigstrut \\
\textit{OFF\_UNIV\_NAME} & \texttt{u}$n$\texttt{@}$j$ & Set of atoms $\{\texttt{A}j, \ldots, \texttt{A}(j + n - 1)\}$\bigstrut \\
\bigstrut\textit{TUPLE\_NAME} & \texttt{P}$j$ & Pair at index $j$ in the pair space associated with the universe\bigstrut \\
\bigstrut & \texttt{T}$n$\texttt{\_}$j$ & $n$-tuple at index $j$ in the $n$-tuple space
associated with the universe ($n \ge 3$)\bigstrut \\
\textit{RELATION\_NAME} & \texttt{s}$j$ & Plain set number $j$\bigstrut \\
& \texttt{s}$j$\texttt{'} & Primed set number $j$\bigstrut \\
& \texttt{r}$j$ & Plain binary relation number $j$\bigstrut \\
& \texttt{r}$j$\texttt{'} & Primed binary relation number $j$\bigstrut \\
& \texttt{m}$n$\texttt{\_}$j$ & Plain $n$-ary multirelation number $j$ ($n \ge 3$)\bigstrut \\
& \texttt{m}$n$\texttt{\_}$j$\texttt{'} & Primed $n$-ary multirelation number $j$ ($n \ge 3$)\bigstrut \\
\textit{VARIABLE\_NAME} & \texttt{S}$j$ & Plain set variable number $j$\bigstrut \\
& \texttt{S}$j$\texttt{'} & Primed set variable number $j$\bigstrut \\
& \texttt{R}$j$ & Plain binary relation variable number $j$\bigstrut \\
& \texttt{R}$j$\texttt{'} & Primed binary relation variable number $j$\bigstrut \\
\bigstrut & \texttt{M}$n$\texttt{\_}$j$ &Plain $n$-ary multirelation variable number
$j$ ($n \ge 3$)\bigstrut \\
\bigstrut & \texttt{M}$n$\texttt{\_}$j$\texttt{'} & Primed $n$-ary multirelation variable 
number $j$ ($n \ge 3$)\bigstrut \\
\textit{TUPLE\_REG} & \texttt{\$A}$j$ & One-tuple register number $j$\bigstrut \\
& \texttt{\$P}$j$ & Pair register number $j$\bigstrut \\
& \texttt{\$T}$n$\texttt{\_}$j$ & $n$-tuple register number $j$ ($n \ge 3$)\bigstrut \\
\textit{TUPLE\_SET\_REG} & \texttt{\$a}$j$ & One-tuple set register number $j$\bigstrut \\
& \texttt{\$p}$j$ & Pair set register number $j$\bigstrut \\
& \texttt{\$t}$n$\texttt{\_}$j$ & $n$-tuple set register number $j$ ($n \ge 3$)\bigstrut \\
\textit{FORMULA\_REG} & \texttt{\$f}$j$ & Formula register number $j$\bigstrut \\
\textit{REL\_EXPR\_REG} & \texttt{\$e}$j$ & Relational expression register number $j$\bigstrut \\
\textit{INT\_EXPR\_REG} & \texttt{\$i}$j$ & Integer expression register number $j$\bigstrut \\
\end{supertabular}
\end{center}

The primed versions of relations and variables are provided for syntactic
convenience. They behave exactly as the unprimed versions, except that they live
in a different namespace.

\subsection{Overall Structure}
\label{overall-structure}

This section presents the overall structure of Kodkodi input files.

\subsubsection{Problems}
\label{problems}
%
$$\small\begin{aligned}
\textit{problems} & \,::=\, \textit{problem}^*
\end{aligned}$$
%
Kodkodi takes a list of ``problems'' as input.

\subsubsection{Problem}
\label{problem}

$$\small\begin{aligned}
\textit{problem} \,::=\, {}& \textit{option}^*\
                  \textit{univ\_spec}\
                  \textit{tuple\_reg\_directive}^*\
                  \textit{bound\_spec}^*\
                  \textit{int\_bound\_spec}^? \\[-.2ex] %% TYPESETTING
                 & \textit{expr\_reg\_directive}^*\ \textit{solve\_directive}
\end{aligned}$$
%
A problem consists of three main parts: a universe specification, a set of bound
specifications, and a Kodkod formula to satisfy supplied in a ``solve''
directive.

Example:

\pre
\ttfamily\small
univ:~u1 \\
bounds s0:~\{A0\} \\
solve all [S0 :~one s0, S1 :~one s0] | S0 = S1;
\post

\subsubsection{Kodkod Options}
\label{kodkod-options}

$$\small\begin{aligned}
\textit{option} \,::=\ {}& \texttt{solver}\; \verb|:|\; \textit{STR\_LITERAL}\, (\verb|,|\; \textit{STR\_LITERAL})^*\; \mid {} \\
           & \texttt{symmetry\_breaking}\; \verb|:|\; \textit{NUM}\; \mid \\
           & \texttt{sharing}\; \verb|:|\; \textit{NUM}\; \mid \\
           & \texttt{bit\_width}\; \verb|:|\; \textit{NUM}\; \mid \\
           & \texttt{skolem\_depth}\; \verb|:|\; \textit{NUM}\; \mid \\
           & \texttt{flatten}\; \verb|:|\; (\verb|true| \mid \verb|false|)\; \mid \\
           & \texttt{delay}\; \verb|:|\; \textit{NUM}
\end{aligned}$$
%
Kodkod supports various options, documented in the \texttt{kodkod.\allowbreak
engine.\allowbreak config.\allowbreak Options} class \cite{kodkod-2009-options}.
The following solvers are supported:

\pre
\ttfamily\small
solver:~"CryptoMiniSat" \\
solver:~"DefaultSAT4J" \\
solver:~"LightSAT4J" \\
solver:~"Lingeling" \\
solver:~"MiniSat" \\
solver:~"MiniSatProver" \\
solver:~"ZChaffMincost" \\
solver:~"SAT4J" "instance" \\
solver:~"External", "executable", "cnf\_file", \\
~~~~~~~~"arg\_1", $\ldots$, "arg\_n" \\
solver:~"ExternalV2", "executable", "temp\_input", "temp\_output", \\
~~~~~~~~"sat\_marker", "var\_marker", "unsat\_marker", "arg\_1", $\ldots$, "arg\_n"
\post

For \verb|"External"|, the optional arguments \verb|"arg_1"|, \ldots,
\verb|"arg_n"| are passed before the input file name. For \verb|"ExternalV2"|,
they are passed after.

The \verb|delay| option specifies a delay (expressed in milliseconds) between
solving a problem and exiting, if the \verb|-exit-on-success| command-line
option is specified. This can be used to grant additional time to other threads
so that they have a chance to finish.

\subsubsection{Universe Specification}
\label{universe-specification}

$$\small\begin{aligned}
\textit{univ\_spec} \,::=\, {} & \verb|univ|\; \verb|:|\; \textit{UNIV\_NAME}
\end{aligned}$$
%
The universe specification fixes the universe's uninterpreted atoms. Kodkodi
requires that the atoms are numbered consecutively from \texttt{A0} to
\texttt{A}$(n - 1)$.

Examples:

\pre
\ttfamily\small
univ:~u2 \\
univ:~u100
\post

\subsubsection{Relation Bound Specifications}
\label{relation-bound-specifications}

$$\small\begin{aligned}
\textit{bound\_spec} \,::=\, {}& \verb|bounds|\; \textit{RELATION\_NAME}\; (\verb|,|\; \textit{RELATION\_NAME})\; \verb|:| \\[-.2ex] %% TYPESETTING
& ( \textit{tuple\_set} \mid \verb|[|\; \textit{tuple\_set}\; \verb|,| \; \textit{tuple\_set}\; \verb|]| )
\end{aligned}$$
%
A relational bound specification gives a lower and an upper bound for the given relations. If only one bound is specified, it is taken as both lower and upper bound. The lower bound must be a subset of the upper bound.

Examples:

\pre
\ttfamily\small
bounds s0:~\{A0\} \\
bounds r2:~[\{\}, \{A0 ..\ A9\} -> \{A10 ..\ A19\}]
\post

\subsubsection{Integer Bound Specification}
\label{integer-bound-specification}

$$\small\begin{aligned}
\textit{int\_bound\_spec} &\,::=\, \verb|int_bounds|\; \verb|:|\; \textit{int\_bound\_seq}\; (\verb|,|\; \textit{int\_bound\_seq})^* \\
\textit{int\_bound\_seq} & \,::=\, [\textit{NUM}\;\verb|:|]\;\verb|[|\; \textit{tuple\_set}\; (\verb|,|\; \textit{tuple\_set})^*\; \verb|]|
\end{aligned}$$
%
An integer bound specification establishes a correspondence between integers and
sets of atoms that represent that integer in relational expressions. The syntax
makes it possible to specify the bounds of consecutive integers in sequence.

Example:

\pre
\ttfamily\small
int\_bounds:~[\{A0\}, \{A1\}], 10:~[\{A2\}, \{A3\}, \{A4\}]
\post

In the above example, \verb|0| is bounded by \verb|{A0}|, \verb|1| is bounded by
\verb|{A1}|, \verb|10| is bounded by \verb|{A2}|, \verb|11| is bounded by
\verb|{A11}|, and \verb|12| is bounded by \verb|{A4}|.

\subsubsection{Solve Directive}
\label{solve-directive}

$$\small\begin{aligned}
\textit{solve\_directive} \,::=\, {}& \verb|solve|\; \textit{formula}\; \verb|;|
\end{aligned}$$
%
The ``solve'' directive tells Kodkod to try to satisfy the given formula.

Example:

\pre
\ttfamily\small
solve all [S0 :~one s0, S1 :~one s0] | !~S0 = S1 => no S0.r0 \& S1.r0
\post

\subsection{Register Directives}
\label{register-directives}

Registers make it possible to use a complex syntactic construct several times
without duplicating it. They also help reduce Kodkod's memory usage and running
time.

\subsubsection{Tuple Register Directives}
\label{tuple-register-directives}

$$\small\begin{aligned}
\textit{tuple\_reg\_directive} \,::=\, {}
    & \textit{TUPLE\_REG}\; \verb|:=|\; \textit{tuple} \mid {} \\
    & \textit{TUPLE\_SET\_REG}\; \verb|:=|\; \textit{tuple\_set}
\end{aligned}$$
%
A tuple register directive assigns a value to a tuple or tuple set register.

Examples:

\pre
\ttfamily\small
\$P0 := [A0, A0] \\
\$P1 := [A1, A1] \\
\$t4\_0 := \{\$P0, \$P1\} -> \{\$P0, \$P1\}
\post

\subsubsection{Expression Register Directives}
\label{expression-register-directives}

$$\small\begin{aligned}
\textit{expr\_reg\_directive} \,::=\, {}
    & \textit{FORMULA\_REG}\; \verb|:=|\; \textit{formula} \mid {} \\
    & \textit{REL\_EXPR\_REG}\; \verb|:=|\; \textit{rel\_expr} \mid {} \\
    & \textit{INT\_EXPR\_REG}\; \verb|:=|\; \textit{int\_expr}
\end{aligned}$$
%
Formulas, relational expressions, and integer expressions can also be assigned
to registers using an expression register directive. An alternative is to use
the \verb|let| binder inside an expression.

Examples:

\pre
\ttfamily\small
\$f0 := all [S0 :~one s0] | s0 in univ \\
\$e5 := (s0 \& s1).r1 + (s0 \& s2).r2 \\
\$i14 := 2 * \#(\$e5) + 1
\post

\subsection{Tuple Language}
\label{tuple-language}

Kodkod supports partial solutions in the form of bounds on relations. The bound
specifications involve tuples and tuple sets.

\subsubsection{Tuples}
\label{tuples}

$$\small\begin{aligned}
\textit{tuple} \,::=\,{}
    & \verb|[|\; \textit{ATOM\_NAME}\; (\verb|,|\; \textit{ATOM\_NAME})^*\; \verb|]| \mid {} \\
    & \textit{ATOM\_NAME} \mid {} \\
    & \textit{TUPLE\_NAME} \mid {} \\
    & \textit{TUPLE\_REG}
\end{aligned}$$
%
An $n$-tuple is normally specified using the syntax \texttt{[A\textit{j}$_1$,
$\ldots$, A\textit{j}$_n$]}. The brackets are optional when $n = 1$.
Alternatively, tuples can be specified using an index in the $n$-tuple space.
For example, given the universe \verb|u10|, the name \verb|P27| refers to the
pair \verb|[A2, A7]|.

Examples:

\pre
\ttfamily\small
[A0, A1, A5, A20] \\
A0 \\
P5 \\
\$P14
\post

\subsubsection{Tuple Sets}
\label{tuple-sets}

$$\small\begin{aligned}
\textit{tuple\_set} \,::=\,{}
    & \textit{tuple\_set}\; (\verb|+| \mid \verb|-|)\; \textit{tuple\_set} \mid \\
    & \textit{tuple\_set}\; \verb|&|\; \textit{tuple\_set} \mid \\
    & \textit{tuple\_set}\; \verb|->|\; \textit{tuple\_set} \mid \\
    & \textit{tuple\_set}\; \verb|[|\; \textit{NUM}\; \verb|]| \mid \\
    & \verb|{|\; \textit{tuple}\; (\verb|,|\; \textit{tuple})^*\; \verb|}| \mid \\
    & \verb|{|\; \textit{tuple}\; \verb|..|\; \textit{tuple}\; \verb|}| \mid \\
    & \verb|{|\; \textit{tuple}\; \verb|#|\; \textit{tuple}\; \verb|}| \mid \\
    & \verb|none| \mid \\
    & \verb|all| \mid \\
    & \textit{UNIV\_NAME} \mid \\
    & \textit{OFF\_UNIV\_NAME} \mid \\
    & \textit{TUPLE\_SET\_REG} \mid \\
    & \verb|(|\; \textit{tuple\_set}\; \verb|)|
\end{aligned}$$
%
Tuple sets can be constructed in several ways. The \verb|+|, \verb|-|, and
\verb|&| operators denote the union, difference, and intersection of two tuple
sets, respectively. The \verb|->| operator denotes the Cartesian product of two
tuple sets. The \verb|[]| operator projects the tuple set onto the given
dimension. Tuple sets can be specified exhaustively by listing all their tuples.
If all the tuples have consecutive indices, the range operator \verb|..| can be
used. Alternatively, if all the tuples occupy a rectangular, cubic, etc., area
in the tuple space, they can be specified by passing the lowest and highest
corner of the area to the \verb|#| operator. Finally, \verb|none| is a synonym
for \verb|{}|, and \verb|all| denotes the complete tuple set (whose arity is
deduced from the context).

Examples:

\pre
\ttfamily\small
\{A1\} \\
\{A1, A2\} -> \{A3, A4\} \\
\{[A1, A2] {.}{.} [A3, A4]\} \\
\{[A1, A2] \# [A3, A4]\} \\
\$p14
\post

\subsubsection{Tuple Set Operator Precedences and Associativities}
\label{tuple-set-operator-precedences-and-associativities}

The operator precedences and associativities are given in the table below. Fully
bracketed operators are not listed.

\begin{center}
\small
\begin{supertabular}{c|l|l|l} %% TYPESETTING
Level & Operator Class & Arity & Associativity \\[.4ex]
\hline
\strut\vphantom{$A^{B^{C^{D}}}$}%
1 & \verb|+|~~~\verb|-| & Binary & Left-associative\bigstrut \\
2 & \verb|&| & Binary & Associative\bigstrut \\
3 & \verb|->| & Binary & Associative\bigstrut \\
4 & \verb|[]| & Binary & Left-associative\bigstrut \\
\end{supertabular}
\end{center}

\subsection{Expression Language}
\label{expression-language}

Kodkod supports three types of expression: Boolean expressions (formulas),
relational expressions, and integer expressions.

\subsubsection{Formulas}
\label{formulas}

$$\small\begin{aligned}
\textit{formula} \,::=\, {}
    & (\verb|all| \mid \verb|some|)\; \textit{decls}\; \verb!|!\; \textit{formula} \mid {} \\
    & \verb|let|\; \textit{assigns}\; \verb!|!\; \textit{formula} \mid {} \\
    & \verb|if|\; \textit{formula}\; \verb|then|\; \textit{formula}\; \verb|else|\; \textit{formula} \mid {} \\
    & \textit{formula}\; \verb!||!\; \textit{formula} \mid {} \\
    & \textit{formula}\; \verb|<=>|\; \textit{formula} \mid {} \\
    & \textit{formula}\; \verb|=>|\; \textit{formula} \mid {} \\
    & \textit{formula}\; \verb|&&|\; \textit{formula} \mid {} \\
    & \verb|!|\; \textit{formula} \mid {} \\
    & \verb|ACYCLIC|\; \verb|(|\; \textit{RELATION\_NAME}\; \verb|)| \mid {} \\
    & \verb|FUNCTION|\; \verb|(|\; \textit{RELATION\_NAME}\; \verb|,|\; \textit{rel\_expr}\; \verb|->|\; (\verb|one| \mid \verb|lone|)\; \textit{rel\_expr}\; \verb|)| \mid {} \\
    & \verb|TOTAL_ORDERING|\; \verb|(|\; \textit{RELATION\_NAME}\; \verb|,|\; \\[-.2ex] %% TYPESETTING
        &\qquad (\textit{UNIV\_NAME} \mid \textit{OFF\_UNIV\_NAME} \mid \textit{RELATION\_NAME})\; \verb|,|\; \\[-.2ex] %% TYPESETTING
        &\qquad (\textit{ATOM\_NAME} \mid \textit{RELATION\_NAME})\;\verb|,|\; \\[-.2ex] %% TYPESETTING
        &\qquad (\textit{ATOM\_NAME} \mid \textit{RELATION\_NAME})\; \verb|)| \mid {} \\
    & \textit{rel\_expr}\; (\verb|in| \mid \verb|=|)\; \textit{rel\_expr} \mid {} \\
    & \textit{int\_expr}\; (\verb|=| \mid \verb|<| \mid \verb|<=| \mid \verb|>| \mid \verb|>=|)\; \textit{int\_expr} \mid {} \\
    & (\verb|no| \mid \verb|lone| \mid \verb|one| \mid \verb|some|)\; \textit{rel\_expr} \mid {} \\
    & \verb|false| \mid {} \\
    & \verb|true| \mid {} \\
    & \textit{FORMULA\_REG} \mid {} \\
    & (\; \textit{formula}\; )
\end{aligned}$$
%
A formula, or Boolean expression, specifies a constraint involving relations and integers.

Example:

\pre
\ttfamily\small
some [S0 :~some s0] | if S0 in s1 then !\$f1 else \$i0 <= \$i1
\post

\subsubsection{Relational Expressions}
\label{relational-expressions}

$$\small\begin{aligned}
\textit{rel\_expr} \,::=\, {}
    & \verb|let|\; \textit{assigns}\; \verb!|!\; \textit{rel\_expr} \mid {} \\
    & \verb|if|\; \textit{formula}\; \verb|then|\; \textit{rel\_expr}\; \verb|else|\; \textit{rel\_expr} \mid {} \\
    & \textit{rel\_expr}\; (\verb|+| \mid \verb|-|)\; \textit{rel\_expr} \mid {} \\
    & \textit{rel\_expr}\; \verb|++|\; \textit{rel\_expr} \mid {} \\
    & \textit{rel\_expr}\; \verb|&|\; \textit{rel\_expr} \mid {} \\
    & \textit{rel\_expr}\; \verb|->|\; \textit{rel\_expr} \mid {} \\
    & \textit{rel\_expr}\; \verb|\|\; \textit{rel\_expr} \mid {} \\
    & \textit{rel\_expr}\; \verb|(|\; \textit{rel\_expr}\; (\verb|,| \textit{rel\_expr})^*\; \verb|)| \mid {} \\
    & \textit{rel\_expr}\; \verb|[|\; \textit{int\_expr}\; (\verb|,|\; \textit{int\_expr})^*\; \verb|]| \mid {} \\
    & \textit{rel\_expr}\; \verb|.|\; \textit{rel\_expr} \mid {} \\
    & (\verb|^| \mid \verb|*| \mid \verb|~|)\; \textit{rel\_expr} \mid {} \\
    & \verb|{|\; \textit{decls}\; \verb!|!\; \textit{formula}\; \verb|}| \mid {} \\
    & (\verb|Bits| \mid \verb|Int|)\; \verb|[|\; \textit{int\_expr} \;\verb|]| \mid {} \\
    & \verb|iden| \mid {} \\
    & \verb|ints| \mid {} \\
    & \verb|none| \mid {} \\
    & \verb|univ| \mid {} \\
    & \textit{ATOM\_NAME} \mid {} \\
    & \textit{UNIV\_NAME} \mid {} \\
    & \textit{OFF\_UNIV\_NAME} \mid {} \\
    & \textit{RELATION\_NAME} \mid {} \\
    & \textit{VARIABLE\_NAME} \mid {} \\
    & \textit{REL\_EXPR\_REG} \mid {} \\
    & \verb|(|\; \textit{rel\_expr}\; \verb|)| 
\end{aligned}$$
%
A relational expression denotes a relation (set, binary relation, or
multirelation). Nearly all operators are identical to those offered by Kodkod,
which in turn are modeled after those provided by Alloy. Notable exceptions are
the conditional expression \verb|if| \ldots \verb|then| \ldots \verb|else| \ldots; the \texttt{\textit{r}} \verb|\| \textit{\textit{s}} operator, which is
a shorthand for \verb|if| \verb|no| \texttt{\textit{r}} \verb|then| \texttt{\textit{s}} \verb|else| \texttt{\textit{r}}; and finally \texttt{\textit{r}(\textit{s}$_1$,} \texttt{$\ldots$,} \texttt{\textit{s}$_n$)},
which is equivalent to
\texttt{\textit{s}$_n$.($\ldots$(\textit{s}$_1$.r)$\ldots$)}.

Example:

\pre
\ttfamily\small
if \#(s0) > 5 then s0.r0 + s1.r1 else none
\post

\subsubsection{Integer Expressions}
\label{integer-expressions}

$$\small\begin{aligned}
\textit{int\_expr} \,::=\,{}
    & \verb|sum|\; \textit{decls}\; \verb!|!\; \textit{int\_expr} \mid {} \\
    & \verb|let|\; \textit{assigns}\; \verb!|!\; \textit{int\_expr} \mid {} \\
    & \verb|if|\; \textit{formula}\; \verb|then|\; \textit{int\_expr}\; \verb|else|\; \textit{int\_expr} \mid {} \\
    & \textit{int\_expr}\; (\verb|<<| \mid \verb|>>| \mid \verb|>>>|)\; \textit{int\_expr} \mid {} \\
    & \textit{int\_expr}\; (\verb|+| \mid \verb|-|)\; \textit{int\_expr} \mid {} \\
    & \textit{int\_expr}\; (\verb|*| \mid \verb|/| \mid \verb|%|)\; \textit{int\_expr} \mid {} \\
    & (\verb|#| \mid \verb|sum|)\; \verb|(|\; \textit{rel\_expr} \verb|)| \mid {} \\
    & \textit{int\_expr}\; \verb!|! \;\textit{int\_expr} \mid {} \\
    & \textit{int\_expr}\; \verb|^| \;\textit{int\_expr} \mid {} \\
    & \textit{int\_expr}\; \verb|&| \;\textit{int\_expr} \mid {} \\
    & (\verb|~| \mid \verb|-| \mid \verb|abs| \mid \verb|sgn|)\; \textit{int\_expr} \mid {} \\
    & \textit{NUM} \mid {} \\
    & \textit{INT\_EXPR\_REG} \mid {} \\
    & \verb|(|\; \textit{int\_expr}\; \verb|)|
\end{aligned}$$
%
An integer expression denotes an integer.

Example:

\pre
\ttfamily\small
(sum [S0 :~one s0] | \#(S0) * (\#(S0) + 1) / 2) \% 10
\post

\subsubsection{Declarations}
\label{declarations}

$$\small\begin{aligned}
\textit{decls} & \,::=\, \verb|[|\; \textit{decl}\; (\verb|,|\; \textit{decl})^*\; \verb|]| \\
\textit{decl} & \,::=\, \textit{VARIABLE\_NAME}\; \verb|:|\; (\verb|no| \mid \verb|lone| \mid \verb|one| \mid \verb|some| \mid \verb|set|)\; \textit{rel\_expr}
\end{aligned}$$
%
The \verb|all|, \verb|some|, and \verb|sum| quantifiers take a list of variable
declarations.

Example:

\pre
\ttfamily\small
[S0 :~set s0, S1 :~one s1]
\post

\subsubsection{Assignments}
\label{assignments}

$$\small\begin{aligned}
\textit{assigns} \,::=\,{} & \verb|[|\; \textit{assign}\; (\verb|,|\; \textit{assign})^*\;\verb|]| \\
\textit{assign} \,::=\,{} & \textit{FORMULA\_REG}\; \verb|:=|\; \textit{formula} \mid {} \\
    & \textit{REL\_EXPR\_REG}\; \verb|:=|\; \textit{rel\_expr} \mid {} \\
    & \textit{INT\_EXPR\_REG}\; \verb|:=|\; \textit{int\_expr}
\end{aligned}$$
%
The \verb|let| binder takes a list of register assignments.

Example:

\pre
\ttfamily\small
[\$f0 := all [S0 :~one s0] | s0 in univ, \$i14 := 2 * \#(\$e5) + 1]
\post

\subsubsection{Operator Precedences and Associativities}
\label{operator-precedences-and-associativities}

The operator precedences and associativities are given in the table below. Fully
bracketed operators are not listed.

\begin{center}
\small
\begin{supertabular}{c|l|l|l} %% TYPESETTING
Level & Operator Class & Arity & Associativity \\[.4ex]
\hline
\strut\vphantom{$A^{B^{C^{D}}}$}%
%1 & \verb|+|~~\verb|-| & Binary & Left-associative\bigstrut \\
1 & \verb!all |   some |   sum |! & Binary/Ternary & Right-associative\bigstrut \\
  & \verb!let |   if then else! & &\bigstrut \\
2 & \verb!||! & Binary & Associative\bigstrut \\
3 & \verb|<=>| & Binary & Associative\bigstrut \\
4 & \verb|=>| & Binary & Right-associative\bigstrut \\
5 & \verb|&&| & Binary & Associative\bigstrut \\
6 & \verb|!| & Unary & N/A\bigstrut \\
7 & \verb|in   =   <   <=   >   >=| & Binary & N/A\bigstrut \\
8 & \verb|no   lone   one   some| & Unary & N/A\bigstrut \\
9 & \verb|<<   >>   >>>| & Binary & Left-associative\bigstrut \\
10 & \verb|+   -| & Binary & Left-associative\bigstrut \\
11 & \verb|*   /   %| & Binary & Left-associative\bigstrut \\
12 & \verb|++| & Binary & Associative\bigstrut \\
13 & \verb!|   ^   &! & Binary & Associative\bigstrut \\
14 & \verb|->| & Binary & Associative\bigstrut \\
15 & \verb|\| & Binary & Associative\bigstrut \\
16 & \verb|(,)| & Binary & Left-associative\bigstrut \\
17 & \verb|[,]| & Binary & Left-associative\bigstrut \\
18 & \verb|.| & Binary & Left-associative\bigstrut \\
19 & \verb|^   *   ~   -   abs   sgn| & Unary & N/A\bigstrut \\
\end{supertabular}
\end{center}

\subsection{Comments}
\label{comments}

Comments may be specified as in \cpp, that is, either as a one line comment
starting with \verb|//| or as a block starting with \verb|/*| and ending with
\verb|*/|.

Examples:

\pre
\ttfamily\small
/* \\
~~Copyright 2009 Gnomovision, Inc. \\
*/ \\[2\smallskipamount]
univ:~u99999  // Don't panic!
\post

\section{Case Study: Sorting Using Alloy and Kodkodi}
\label{case-study-sorting-using-alloy-and-kodkodi}

Although Kodkodi's syntax is similar to Alloy's, there are a few important
conceptual differences. Consider the following Alloy specification of integer
sorting:

\pre
\ttfamily\small
abstract sig IntSeq \{ \\
~~ints :~seq Int \\
\} \\[2\smallskipamount]
pred isSorted [s :~IntSeq] \{ \\
~~all i :~s.ints.inds - s.ints.lastIdx | s.ints[i] <= s.ints[i + 1] \\
\} \\
pred isPermutation [pre, post :~IntSeq] \{ \\
~~all p :~Int | \#\{pre.ints.p\} = \#\{post.ints.p\} \\
\} \\[2\smallskipamount]
one sig Pre extends IntSeq \{\} \\
one sig Post extends IntSeq \{\} \\[2\smallskipamount]
fact \{ \\
~~Pre.ints[0] = 7 \&\& Pre.ints[1] = 2 \&\& \\
~~Pre.ints[2] = 4 \&\& Pre.ints[3] = 3 \&\& \\
~~Pre.ints[4] = 3 \&\& Pre.ints[5] = 8 \&\& \\
~~Pre.ints[6] = 5 \&\& Pre.ints[7] = 20 \&\& \\
~~Pre.ints[8] = 18 \&\& Pre.ints[9] = 1 \&\& \\
~~Pre.ints[10] = 10 \&\& Pre.ints[11] = 5 \&\& \\
~~Pre.ints[12] = 7 \&\& Pre.ints[13] = 12 \&\& \\
~~Pre.ints[14] = 2 \&\& Pre.ints[15] = 19 \&\& \\
~~Pre.ints[16] = 15 \&\& Pre.ints[17] = 13 \&\& \\
~~Pre.ints[18] = 11 \&\& Pre.ints[19] = 4 \\
\} \\[2\smallskipamount]
run \{ Pre.isPerm[Post] \&\& Post.isSorted \} for 20 seq, 6 int
\post

There are two main approaches to representing this in Kodkod:

\begin{enum}
\item[1.] \textsl{We could tell the Alloy Analyzer to generate Kodkod-based Java
code, call \texttt{\textsl{toString()}} on the abstract syntax tree, and fiddle
a little bit with the output to make it comply with Kodkodi's input syntax.}

From an Alloy specification, we can generate Java code by chosing ``Output
Kodkod to file'' as the SAT Solver in the Alloy Analyzer's ``Options'' menu.

Unfortunately, for the example above, the generated code is too large for the
Java compiler, which simply bails out. In general, we would need to rename the
atoms and relations so that they follow Kodkodi's strict naming conventions and
change a few syntactic items.

\item[2.]
\textsl{We ignore the Alloy model and start from scratch in Kodkodi.}

This gives a specification like the following:

\pre
\ttfamily\small
solver:~"MiniSat" \\
bit\_width:~6 \\
univ:~u21 \\
bounds r0 /* Pre.ints */: \\
\{[A0, A7], [A1, A2], [A2, A4], [A3, A3], [A4, A3], [A5, A8], [A6, A5],
 [A7, A20], [A8, A18], [A9, A1], [A10, A10], [A11, A5], [A12, A7], [A13, A12],
 [A14, A2], [A15, A19], [A16, A15], [A17, A13], [A18, A11], [A19, A4]\} \\
bounds r1 /* Post.ints */:~[\{\}, u20->u21] \\
int\_bounds:~[\{A0\}, \{A1\}, \{A2\}, \{A3\}, \{A4\}, \{A5\}, \{A6\}, \{A7\},
\{A8\}, \{A9\}, \{A10\}, \{A11\}, \{A12\}, \{A13\}, \{A14\}, \{A15\}, \{A16\},
\{A17\}, \{A18\}, \{A19\}, \{A20\}] \\
solve FUNCTION(r1, u20->one u21) \\
\&\& (all [S0 :~one univ] | \#(r1.S0) = \#(r0.S0)) \\
\&\& (all [S0 :~one u19] | sum(S0.r1) <= sum(Int[sum(S0) + 1].r1));
\post

The first two lines,

\pre
\ttfamily\small
solver:~"MiniSat" \\
bit\_width:~6
\post

are configuration options. Then we specify that the universe should consist of
exactly 21 atoms:

\pre
\ttfamily\small
univ:~u21
\post

The atoms are called \texttt{A0} to \texttt{A20}. Next, we specify the values
for the \texttt{Pre.ints} relation as a Kodkod bound:

\pre
\ttfamily\small
bounds r0 /* Pre.ints */: \\
\{[A0, A7], [A1, A2], [A2, A4], [A3, A3], [A4, A3], [A5, A8], [A6, A5],
 [A7, A20], [A8, A18], [A9, A1], [A10, A10], [A11, A5], [A12, A7], [A13, A12],
 [A14, A2], [A15, A19], [A16, A15], [A17, A13], [A18, A11], [A19, A4]\}
\post

In Kodkodi, all binary relations must be called \texttt{r$j$}, where $j$ is a
natural number. The comment is there to remind us that \texttt{r0} corresponds
to \texttt{Pre.ints} in the Alloy specification.

\pre
\ttfamily\small
bounds r1 /* Post.ints */:~[\{\}, u20->u21]
\post

For \texttt{Post.ints}, we specify the empty set \texttt{\{\}} as the lower
bound and the Cartesian product \texttt{\{A0 {..}~A19\}->\{A0 {.{.}}~A20\}} as
the upper bound.

\pre
\ttfamily\small
int\_bounds:~[\{A0\}, \{A1\}, \{A2\}, \{A3\}, \{A4\}, \{A5\}, \{A6\}, \{A7\},
\{A8\}, \{A9\}, \{A10\}, \{A11\}, \{A12\}, \{A13\}, \{A14\}, \{A15\}, \{A16\},
\{A17\}, \{A18\}, \{A19\}, \{A20\}]
\post

Since we need the integers for addition, we must associate atoms with the
integers we need. Here we simply let \texttt{A0} represent \texttt{0},
\texttt{A1} represent \texttt{1}, and so on.

\pre
\ttfamily\small
solve FUNCTION(r1, u20->one u21) \\
\&\& (all [S0 :~one univ] | \#(r1.S0) = \#(r0.S0)) \\
\&\& (all [S0 :~one u19] | sum(S0.r1) <= sum(Int[sum(S0) + 1].r1));
\post

Finally, we specify the formula to solve. The first line ensures that
\texttt{r1} (i.e., \texttt{Post.ints}) is a function rather than an arbitrary
relation. The second and third lines are adapted directly from the Alloy
specification.

Sorting [7, 2, 4, 3, 3, 8, 5, 20, 18, 1, 10, 5, 7, 12, 2, 19, 15, 13, 11, 4]
should give [1, 2, 2, 3, 3, 4, 4, 5, 5, 7, 7, 8, 10, 11, 12, 13, 15, 18, 19,
20]. Let us verify that this is the case by running Kodkodi:

\pre
\ttfamily\small\slshape
*** PROBLEM 1 *** \\[2\smallskipamount]
---OUTCOME--- \\
SATISFIABLE \\[2\smallskipamount]
---INSTANCE--- \\
relations:~\{r0=[[A0, A7], [A1, A2], [A2, A4], [A3, A3], [A4, A3], [A5, A8],
[A6, A5], [A7, A20], [A8, A18], [A9, A1], [A10, A10], [A11, A5], [A12, A7],
[A13, A12], [A14, A2], [A15, A19], [A16, A15], [A17, A13], [A18, A11], [A19,
A4]], r1=[[A0, A1], [A1, A2], [A2, A2], [A3, A3], [A4, A3], [A5, A4], [A6, A4],
[A7, A5], [A8, A5], [A9, A7], [A10, A7], [A11, A8], [A12, A10], [A13, A11],
[A14, A12], [A15, A13], [A16, A15], [A17, A18], [A18, A19], [A19, A20]]\}
\\[2\smallskipamount]
---STATS--- \\
p cnf 8166 29484 \\
primary variables:~420 \\
parsing time:~72 ms \\
translation time:~359 ms \\
solving time:~434 ms
\post

The result is correct.
\end{enum}

\section{Known Bugs and Limitations}
\label{known-bugs-and-limitations}

Here are the known bugs and limitations in Kodkodi at the time of writing:

\begin{enum}
\item[$\bullet$] The \texttt{-server} command-line option, which makes Kodkodi
run as a TCP server, is limited to a single connection. Furthermore, any error
occurring when processing one problem breaks the connection.
\end{enum}

\let\em=\sl
\bibliography{kodkodi}{}
\bibliographystyle{abbrv}

\end{document}
